\chapter{Approaches to Time Series compression}
\section{Storing aggregates}
Storing aggregates is probably the most common approach to deal with large time series.
The idea is to reduce the temporal resolution of data by storing aggregates of values within
some predefined time buckets. Once values have been aggregated to a lower temporal resolution,
they can be deleted, provided that the initial temporal resolution is not needed anymore.
As a practical example, let us consider an application in which we record the time delays
accumulated by trains. The following series represent the daily delay of a specific train on
a certain day in minutes.
\begin{table}[!htbp]
\centering
\begin{tabular}{l|l|l|l|l|l|l}
\textbf{Day 1} & \textbf{Day 2} & \textbf{Day 3} & \textbf{Day 4} & \textbf{Day 5} & \textbf{Day 6} & \textbf{Day 7} \\
\hline
20 & 20 & 10 & 50 & 40 & 40 & 30 \\                
\end{tabular}
\caption{Daily delay of a specific train of a certain day in minutes.}
\label{tab:daily_delay}
\end{table}

After each week passes, we want to aggregate the previous week’s daily data. With reference
to our example above, we can delete the daily values and replace them by their sum
(210 minutes). By also storing the count, we can compute the average daily during that week
(210 / 7 = 30 minutes).
This technique, often called downsampling, is so simple and effective that most time series
databases offer it out of the box. For example, OpenTSDB, an open-source time series database,
offers the possibility to query downsampled data \cite{A2019Downsample}. This is useful when
one wants to plot data keeping the density at a level that can be easily understandable by
users. Admittedly, this feature does not hold any savings in terms of storage, as it is
performed only at query time, but it is still interesting for network bound applications.
Since version 2.4 OpenTSDB offers the possibility to store the result of the downsampler
\cite{A2019Rollup}. In this way, when querying for a long time span, there is no need to wait for the
downsampled data to be computed.
InfluxDB, another popular open-source database, offers the possibility to downsample old
data using continuous queries and retention policy \cite{A2016Downsampling}. Continuous queries
are queries that are run automatically and periodically within a database. Retention policy
defines for how long InfluxDB keeps the data. By using these concepts together, one can use
a continuous query to aggregate old data into a lower temporal resolution and delete the
original data automatically by setting an appropriate retention policy.

\section{Lossless compression}
Lossless compression is a method of data compression that allows to reconstruct the original
data from the compressed data \cite{WikipediaContributors2019Data}. When applied to time series, it is used when we are
interested in retaining the original data but we want to reduce storage space. In this
section, we will present two different algorithms that have been devised to solve different
problems. The first algorithm was created at Facebook to face the huge amount of data
gathered by monitoring their systems. This algorithm has allowed them to store all data
in memory, thus improving considerably reading and writing performance.
The second algorithm we present, called Sprintz, was created to limit the energy consumption
of IoT devices. By sending compressed data, the network interface use is reduced, thus
obtaining considerable power savings.
