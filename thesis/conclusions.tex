\chapter*{Conclusions}
This thesis aimed to identify the principal approaches and algorithms for time series data compression.
We have identified four main approaches that are used in real-world applications and databases.
We then focused on lossless compression algorithms by creating benchmarks to evaluate how Gorilla,
an algorithm specifically tailored for infrastructure monitoring time series data, performs against
state of the art general purpose algorithms. 
These benchmarks clearly illustrate how Gorilla can outperform the general purpose algorithm, but it
also shows how this is highly dependent on the correlation among the values in the time series.
Based on these results, we suggest to first analyze which kind of data one is dealing with before choosing
a compression algorithm. Our benchmarks showed that the best results are obtained with time series data sets where
the same values are repeated in neighboring data points.

We should note that Gorilla compression deals only with univariate time series, which can limit
its usefulness. In monitoring applications, usually different metrics are collected for the
same entity at regular intervals. As an example, one could measure disk space, CPU usage and memory usage
for and store them as a multivariate time series. Gorilla would require us to split such time series into
three different univariate time series, thus causing the same timestamps to be repeated three times.
Adding some sort of bit packing, as Sprintz does, would allow us to use Gorilla also for multivariate time series.
One limit of our benchmarks is that we did not consider compression and compression speed.
This might be interesting especially when dealing with database queries and inserts. On the one hand,
data compression increases the information density of transferred data, thus
optimizing the usage of disk bandwidth. On the other hand, the decompression overhead might outweigh the potential
benefits of compression. One recent work has found that query times were improved
considerably after using compression \cite{burman2017implementing}.

In conclusion, this thesis provides an overview of the problem of time series compression and offers some insights
into the compression performance of Gorilla and other general purposes lossless algorithms. Future research should expand
on these findings by including other compression algorithms and by taking account compression and
decompression speed.

