\chapter*{Conclusions}
This thesis aimed to identify the principal approaches and algorithms for time-series data compression.
We have identified four main approaches that are used in real-world applications and databases.
We then created a benchmark to evaluate how Gorilla, an algorithm specifically tailored for infrastructure
monitoring time-series data, performs against state of the art general purposes algorithms. 
These benchmark clearly illustrates how Gorilla can outperform the general purposes algorithm, but it
also shows how this is highly dependent on the correlation between values in the time series.
Based on these results, it is vital to understand the type of data one is dealing with. Our benchmarks
showed that the best results are obtained with time-series data sets whereby the same values are repeated
in neighboring data points. Future research should consider the speed of compression and decompression of
these algorithms. This might be interesting especially when dealing with database queries. On the one hand,
having less data stored might speed up the queries. On the other hand, decompression speed should also
be taken into account.